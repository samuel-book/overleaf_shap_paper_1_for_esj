\renewcommand{\thefootnote}{\alph{footnote}} % Use letters for footnotes

\section{Methods}

Note: further details of methods may be found in the online appendix. 

%%%%%%%%%%%%%%%%%%%%%%%%%%%%%%%%%%%%%%%%%%%%%%%%%%%%%%%%%%%%%%%%%%%%%%%%

\subsection{Data}

Data were retrieved for 246,676 emergency stroke admissions to acute stroke teams in England and Wales for the years 2016 - 2018, obtained from the Sentinel Stroke National Audit Programme (SSNAP). Data fields were provided for the hyper-acute phase of the stroke pathway, up to and including our target feature: \emph{receive thrombolysis}. Of these patients, 88,928 arrived within 4 hours of known stroke onset, and were used in this modelling study. The data included 132 acute stroke hospitals.

%%%%%%%%%%%%%%%%%%%%%%%%%%%%%%%%%%%%%%%%%%%%%%%%%%%%%%%%%%%%%%%%%%%%%%%%

\subsection{Machine learning models (to predict thrombolysis use)}

We used an \emph{eXtreme Gradient Boosting model \cite{chen_xgboost_2016}} (`XGBoost') to predict the probability of use of thrombolysis for each patient from their other feature values.

%%%%%%%%%%%%%%%%%%%%%%%%%%%%%%%%%%%%%%%%%%%%%%%%%%%%%%%%%%%%%%%%%%%%%%%%

\subsubsection{Feature selection}

In order to simplify the model (for enhanced explainability) we selected those features most predictive of thrombolysis use. Features were selected by forward-feature selection, identifying one feature at a time that led to the greatest improvement in accuracy as measured by Receiver Operating Characteristic (ROC) Area Under Curve (AUC). We repeated this process to identify the top 25 features, and used these results to identify the number of features to include in our machine learning models.

%%%%%%%%%%%%%%%%%%%%%%%%%%%%%%%%%%%%%%%%%%%%%%%%%%%%%%%%%%%%%%%%%%%%%%%%

\subsubsection{Model accuracy}

Model accuracy, ROC AUC, sensitivity and specificity were measured using stratified 5-fold cross validation. Predicted thrombolysis rate was compared with actual thrombolysis use at hospital-level.

%%%%%%%%%%%%%%%%%%%%%%%%%%%%%%%%%%%%%%%%%%%%%%%%%%%%%%%%%%%%%%%%%%%%%%%%

\subsubsection{Machine learning models}

For the different analysis included in this paper, we trained three XGBoost models:
\begin{enumerate}
    \item \emph{K-fold model}: A 5-fold train-test cross validation used to  test the accuracy of the model, and to test reproducibility of SHAP values.
       
    \item \emph{All data model}: A single model trained on all patients, use to investigate the relationship between feature values and predictions.
    
    \item \emph{10k holdout model}: A model trained all data apart from a 10k hold-out set. This model is used to mimic a 10k cohort of patients that attends all hospitals (by changing the hospital encoding) to further investigate variation in thrombolysis decision-making between hospitals.
\end{enumerate}

%%%%%%%%%%%%%%%%%%%%%%%%%%%%%%%%%%%%%%%%%%%%%%%%%%%%%%%%%%%%%%%%%%%%%%%%
\subsection{SHapley Additive exPlanation (SHAP) values}

We sought to make our models explainable using SHAP values (calculated using the SHAP library \cite{lundberg_unified_2017}). SHAP provides a measure of the contribution of each feature value to the final predicted probability of receiving thrombolysis for that individual. The SHAP values for each feature are comprised of the feature’s main effect (the effect of that feature in isolation) and all of the pairwise interaction effects with each of the other features. SHAP values provide the influence of each feature as the change in log-odds of receiving thrombolysis (SHAP values expressed as log-odds are additive).

%%%%%%%%%%%%%%%%%%%%%%%%%%%%%%%%%%%%%%%%%%%%%%%%%%%%%%%%%%%%%%%%%%%%%%%%
\subsection{The relationship between feature values and the odds of receiving thrombolysis}

For each feature, we examined the relationship between feature values and their corresponding SHAP values (we used values from the \emph{all data} model).

%%%%%%%%%%%%%%%%%%%%%%%%%%%%%%%%%%%%%%%%%%%%%%%%%%%%%%%%%%%%%%%%%%%%%%%%
\subsection{Investigating how the identity of a hospital influences thrombolysis rate}

For each hospital we compared the mean SHAP main effect value for the hospital attended (using values from the \emph{all data} model) with the hospitals observed thrombolysis use.

To reveal the variation in thrombolysis rate due to hospital, rather than patient mix, we also compared the mean hospital attended SHAP main effect value for the identical 10k patient cohort attending each hospital, with the hospitals' predicted thrombolysis use for this 10k patient cohort (we used values from the \emph{10k holdout} model).

%%%%%%%%%%%%%%%%%%%%%%%%%%%%%%%%%%%%%%%%%%%%%%%%%%%%%%%%%%%%%%%%%%%%%%%%

\subsection{Investigating how patient populations and hospital identity and processes influences thrombolysis rate}

The 10 features in the model can be classified into two subsets: 1) `patient descriptive features' (features that describe the patients characteristics), and 2) `hospital descriptive features' (features that describe the hospital’s identity or processes). To analyse the influence that each subset of features has on the thromoblysis rate, we calculated the `subset SHAP value' for each feature, which only includes the components of its SHAP value that contain effects from the features in the same subset. This is expressed as the sum of the main effect and the interaction effects with the other features in the same subset. Multiple regression models were then fitted to the mean subset SHAP values.

%%%%%%%%%%%%%%%%%%%%%%%%%%%%%%%%%%%%%%%%%%%%%%%%%%%%%%%%%%%%%%%%%%%%%%%%

\subsection{Variation in hospital thrombolysis use for patient subgroups}

Informed by the SHAP values, we analysed the observed and predicted use of thrombolysis in eleven subgroups of patients: one subgroup for `ideally' thrombolysable patients, nine `sub-optimal' thrombolysable patient subgroups (one subgroup per feature), and one subgroup with two sub-optimal features The eleven patient subgroups were defined as:

\begin{enumerate}
  \item An \emph{`ideally'} thrombolysable patient:
  \begin{itemize}
    \setlength\itemsep{-2mm}
    \item NIHSS in range 10-25
    \item Arrival-to-scan time $<$30 minutes
    \item Stroke caused by infarction
    \item Precise stroke onset time known
    \item No pre-stroke disability (mRS 0)
    \item Not taking atrial fibrillation anticoagulants
    \item Onset-to-arrival time $<$90 minutes
    \item Age $<$80
    \item Onset not during sleep
  \end{itemize}
  \item NIHSS $<$5
  \item Imprecisely onset time
  \item Existing pre-stroke disability (mRS $>$2)
  \item Haemorrhagic stroke
  \item 60-90 minutes arrival-to-scan time
  \item Patients using AF anticoagulants
  \item 150-180 minutes onset-to-arrival time
  \item Onset during sleep
  \item Age 80+
  \item NIHSS $<$ 5 \emph{and} with estimated stroke onset time
\end{enumerate}

The observed thrombolysis use for each subgroup at each hospital was taken from the SSNAP dataset. In order to further reveal the variation in thrombolysis that was due to hospital decision-making we predicted thrombolysis use for the same patient subgroups at each hospital by using the \emph{10k holdout} model.

%%%%%%%%%%%%%%%%%%%%%%%%%%%%%%%%%%%%%%%%%%%%%%%%%%%%%%%%%%%%%%%%%%%%%%%%
