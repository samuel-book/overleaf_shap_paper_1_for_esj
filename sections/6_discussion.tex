\section{Discussion}

We have built on our previous work to predict thrombolysis use from patient level data, by creating an \emph{explainable machine learning model} which maintains the high accuracy that we previously achieved (85\%) \cite{allen_use_2022}. Predicted thrombolysis use at each hospital also very closely matched observed thrombolysis use. The SSNAP registry data used therefore appears to contain most of the information used to make thrombolysis decisions in clinical practice, and can explain the very large majority of between-hospital variation in thrombolysis use.

In general, using SHAP values to uncover the relationship between patient characteristics and the probability of receiving thrombolysis, we found that the probability of receiving thrombolysis fell with increasing arrival-to-scan times, was dependent on stroke severity with the probability of receiving thrombolysis being highest between NIHSS 10 and 25, was lower when onset time was estimated rather than known precisely, and fell with increasing disability prior to stroke. These patterns are similar to the observations of a discrete choice experiment with hypothetical patients \cite{de_brun_factors_2018}, but in our study we confirm these patterns in actual use of thrombolysis, can be quantitative about the effect, and we add the importance of time-to-scan and whether an onset time is known precisely. 

Hospital SHAP values correlated very closely with the predicted use of thrombolysis in a 10k cohort of patients, confirming that the hospital SHAP main effect value provides a measure of the predisposition of a hospital to use thrombolysis. We found that hospital identity and processes explained 74\% of the variance in observed thrombolysis for patients arriving in time to receive thrombolysis.

After observing the general patterns that exist in the use of thrombolysis, we created a subgroup of patients reflecting what appeared to be an \emph{ideal} candidate for thrombolysis, and also a subgroup per feature where we expected to see lower use of thrombolysis. Observed thrombolysis in these groups reflected the patterns identified by the SHAP analysis. For the \emph{ideal} candidates of thrombolysis, half of stroke units would give thrombolysis to at least 90\% of these patients, but some units gave it to significantly fewer patients. Use of thrombolysis in the other subgroups of patients was, as expected, lower, but use also varied significantly between hospitals. Hospitals have different levels of tolerance for non-ideal patient characteristics. These patterns, of lower but varying use, were repeated with expected use of thrombolysis in the same 10k patient cohort of patients.

This novel analysis examines and aids understanding of between-hospital variation in clinical decision making in the acute stroke setting. The use of large datasets such as SSNAP to understand sources of variation in clinical practice between large number of acute stroke centres across the UK presents a unique opportunity to understand the specific influences behind the significant residual between-hospital variation in thrombolysis use. In particular, it allows national quality improvement projects such as SSNAP to counter one of the most common objections raised to comparative audit: that the patients presenting to any one particular site are in some way unique, thereby accounting for most of the variation in clinical quality between that site and all the others. Although the patient population does vary between hospitals, and will contribute to the thrombolysis use achievable by an individual hospital, the majority of between-hospital variation can be explained by hospital-level rather than patient-level factors.

It is disappointing that even though this disability-saving treatment was first licensed for approval for use over 20 years ago, it is still subject to such large variation in clinical judgement or opinion regarding the selection of patients most appropriate for use. In our previous work \cite{allen_using_2022}, we have shown that increasing the uptake of thrombolysis through the administration of treatment to more patients and sooner after stroke, offers the prospect of more than doubling the proportion of patients after stroke who are left with little or no disability (mRS 0 or 1). At a time when there is an appropriate focus of effort on expanding the use of endovascular therapy in acute ischaemic stroke, it is sobering to consider how much population benefit there still remains to accrue from the fullest possible implementation of a cheaper technology that has been available for over 20 years. Far greater scrutiny of such residual variation in clinical practice is clearly warranted, given the extent to which it appears to be acting as a barrier to successful implementation. Recent studies have highlighted that clinicians can be reluctant to modify their behaviour in response to audit and feedback when it is not seen to be clinically meaningful, recent or reliable \cite{bekker_give_2022}, so the full potential of audit and feedback is not realised \cite{foy_revitalising_2020} despite the evidence of a beneficial effect especially when baseline performance is low \cite{ivers_audit_2012}. The development of bespoke, individualised feedback (at least at hospital level) based on actual and recent activity may increase the impact of efforts at data-driven quality improvement targeted at increasing overall uptake of thrombolysis through reducing variation.

\subsection{Limitations}

This machine learning study is necessarily limited to data collected for the national stroke audit. Though we have high accuracy, and can identify clear patterns of use of thrombolysis, the data will not be sufficient to provide a decision-support tool or to review decision-making at an individual patient level. Nor is it a causal model. We may also be missing information that could otherwise have improved the accuracy still further. The model has high accuracy and can identify clear patterns, suggesting the capability to identify and characterise a centre's culture in the use of thrombolysis, but we do not identify variation in thrombolysis between individual clinicians in the same hospital.

We acknowledge that not all countries have a national stroke audit dataset, however we hope that this paper helps to demonstrate what type of analysis can be done should resources be allocated to collect their national data.